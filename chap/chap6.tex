\chapter{总结与展望}
\section{本文工作总结}
本文阐述了基于深度学习的推荐系统相关的研究工作,主要内容包括:
基于原始信息的多层感知器模型,基于用户兴趣漂移的长短期兴趣模型,以及两种模型驱动的电影推荐系统。

\subsubsection{基于原始信息的多层感知器模型}
本文提出了基于原始信息的多层感知器模型,作为深度学习在推荐系统领域应用的初次尝试。
通过将原始的用户表示和电影表示由输入层导入至多层感知器中,通过隐藏层的抽象和归纳,
在输出层得到待预测评分数据。同时考虑到用户表示和电影表示之间的相似度和评分数据呈正相关关系,
本文还额外将用户表示和电影表示的差值加入到输入层中,进一步提升了预测准确度。

\subsubsection{基于用户兴趣漂移的长短期兴趣模型}
本文提出了基于用户兴趣漂移的长短期兴趣模型,可以有效地刻画用户个人偏好会随时间流逝发生变化的特征,
可以有效提升推荐系统的性能。本文首先在协同过滤算法的原始假设的基础上,提出了用户的长期兴趣和短期兴趣的定义,
以及由此衍生的两个扩展假设。然后基于这两个扩展假设,利用神经网络结构对用户表示和物品表示进行有效地学习。
紧接着,在基于特征的协同过滤框架中融合学习出的用户表示和物品表示,从而完成评分预测任务。
最后,本文在两类真实的电影评分数据集上进行了实验,证明了提出方法的有效性,并和经典方法在多个指标上进行了对比,
体现了本文方法的优越性。

\subsubsection{长短期兴趣模型驱动的电影推荐系统}
本文实现了长短期兴趣模型驱动的电影推荐系统MovieRec。MovieRec可以根据用户的历史电影评分记录,
为平台中的每位用户和每部电影生成对应的特征表示,并基于此进行评分预测,
具有较高预测评分的电影将会以列表是形式推送给相关用户,用户可以通过对推荐结果进行点赞或者差评向平台进行反馈,
以帮助推荐系统在未来时间可以进行更高的推荐。MovieRec还实现了基于其他推荐算法的推荐结果,
用户可以进行直接对比不同推荐算法之间的差异和优劣性。与此同时,MovieRec还提供了用户观影时间线功能,
用户可以查看自己的历史评分电影,并且可以查看任意电影之间的相似度,直观地感受自己长时间观影过程中兴趣漂移。

\section{未来工作展望}
本文在深度学习在推荐系统中的应用进行了大量的探索,并取得了一定的研究成果,
但推荐任务仍然有较大的改进空间,展望今后的工作,主要包括以下几点扩展方向:

\begin{enumerate}
\item \textbf{评分数值的利用}\\
在本文提出的长短期兴趣模型中,用户观影时间线上的电影都可以绑定到一个对应的评分数值,
但是我们并没有加以有效的利用。因此如何将评分数值融入到用户表示和电影表示的生成过程中,
进一步提升表示中所包含的信息量,是将来的研究方向之一。

\item \textbf{辅助信息的利用}\\
辅助信息在推荐系统中扮演着十分重要的角色,常常被用来缓解冷启动带来的推荐困难问题。
长短期兴趣模型可以理解为对时间辅助信息进行利用,其他的例如用户属性和电影属性在本文中并没有被加以利用。
未来的研究工作中,我们也会致力于提出融合模型,利用辅助信息进一步提升推荐性能。

\item \textbf{端到端的学习过程}\\
本文提出的长短期兴趣模型主要分成两个阶段,第一阶段生成用户表示和电影表示,第二阶段生成评分预测。
如何将这两个阶段变成一个整体,形成端到端的学习过程,让评分误差直接反馈给用户表示和电影表示,
也是未来的工作之一。

\end{enumerate}


