\chapter{相关工作}
\section{推荐系统分类}
目前存在的推荐系统可以大致分为三个类别:基于内容的推荐系统(只使用用户画像信息和物品描述信息),
基于协同过滤的推荐系统(只使用评分数据信息)和混合的推荐系统(同时使用两种信息)。

\subsection{基于内容的推荐系统}
基于内容的推荐系统是最早被使用的推荐算法,它根据用户过去喜欢的物品,为用户推荐和它过去喜欢的物品相似的物品。
例如,一个推荐电影的系统可以根据某个用户之前喜欢看喜剧电影而为他推荐新出品的喜剧电影。
该算法的核心假设是``一个用户可能会喜欢和他曾经喜欢过的物品相似的物品'',
这里的相似的物品是通过物品的内容属性来确定的。在不同的推荐场景中,物品的内容属性往往是不同的。
电影推荐中,被推荐的物品是电影,其属性包括导演、演员、风格和发布年份等。
音乐推荐中,被推荐的物品是音乐,其属性包括歌手、曲风、语种和音乐时长等。
商品推荐中,被推荐的物品是商品,其属性包括种类、价格、销量和上市时间等。

%优缺点

基于内容的推荐系统的过程一般包括以下三个步骤:
\begin{enumerate}
\item \textbf{物品建模:} 通过物品的内容属性,为每个物品抽取一些特征,来表示该物品。
\item \textbf{用户建模:} 利用用户曾经喜欢及不喜欢的物品的特征信息,加权整合出该用户的特征信息,来表示该用户。
\item \textbf{推荐生成:} 通过比较用户模型和候选物品模型的相关程度,为该用户推荐一组相关度最大的物品。
\end{enumerate}

其中物品的内容属性主要分为两类:结构化的属性和非结构化的属性。
所谓结构化的属性就是这个属性的意义比较明确,其取值限定在某个范围内,
而非结构化的属性意义比较模糊,取值也没有限制,难以直接使用。
以电影推荐为例,电影的导演,演员、风格和发布年份都是结构化属性,而电影的剧情介绍等文本信息就属于非结构化属性。
对于结构化信息,我们可以直接将其加到物品特征向量中,而对于非结构化信息,
一个常见的方法是将其转化为向量空间模型(Vector Space Model),
例如我们可以将电影的剧情介绍的文本信息转化为一个长度为词汇表大小的向量,
该向量的每一个维度表示对应词语的权重,TF-IDF是一个常见的权重表示方法,其公式如下:
\begin{equation}
\begin{split}
TF\textrm{-}IDF(t_i, d_j) = TF(t_i, d_j) \cdot \log{ \frac{N}{n_i} }
\end{split}
\end{equation}

其中$TF(t_i, d_j)$是第$i$个词语在第$j$篇文档中出现的次数,$N$是所有文档的数量,$n_i$是包含第$i$个词语的文档数量。
通常我们需要进行归一化以便将不同文档的向量统一到一个数量级上,其公式如下:
\begin{equation}
\begin{split}
w(i, j) = \frac{ TF\textrm{-}IDF(t_i, d_j) }{ \sqrt{\sum_{k=1}^{T}{TF\textrm{-}IDF(t_k, d_j)^2}} } 
\end{split}
\end{equation}

其中$T$代表所有词语的数量。学术界提出了多种计算用户模型和物品模型相似度的算法,
其中最著名的就是cosine相似度,其公式如下:
\begin{equation}
\begin{split}
sim(d_i, d_j) = \frac{ \sum_k{ w(k, i) \cdot w(k, j) } }
{ \sqrt{\sum_{k}{w(k, i)^2}} \cdot \sqrt{\sum_{k}{w(k, j)^2}} } 
\end{split}
\end{equation}

当然,真正的推荐系统中使用的策略往往可以很复杂,比如在用户建模时考虑时间因素,
计算不同时间段内的用户模型,从而发现用户兴趣在历史数据上表现出的偏好变化。
又比如再推荐生成的过程中,使用决策树、支持向量机和神经网络等更高级的模型,
这些方法最核心的环节就是利用用户模型和物品模型之间的相似度进行计算,
因为基于内容的推荐系统不是本文研究的重点,所以在此不加赘述。

\subsection{基于协同过滤的推荐系统}


\subsection{混合的推荐系统}
