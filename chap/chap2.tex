\chapter{相关工作}
\section{推荐系统的问题定义}
推荐系统的应用场景繁多,被推荐的对象也各式各样,为了便于研究推荐方法,
一系列的评测任务被提出,其中最为著名的就是电影评分预测任务。
该任务在数学形式上可以表达为,给定$\mathbf{N}$个用户和$\mathbf{M}$个物品,
评分$r_{ij}$代表第$i$个用户给第$j$个物品的评分数据,
数值的高低对应用户对该物品的兴趣程度。
在真实世界的应用场景中,用户通常只会评价一小部分物品(相对于整个物品集合),
因此,这些评分数据构成了一个巨大的稀疏矩阵$\mathbf{R} \in \mathbb{R}^{N \times M}$,
推荐系统的目标就是去预测这些缺失的评分数据。

\section{推荐系统的分类}
目前存在的推荐系统可以大致分为三个类别:基于内容的推荐系统(只使用用户画像信息和物品描述信息),
基于协同过滤的推荐系统(只使用评分数据信息)和混合的推荐系统(同时使用两种信息)。

\subsection{基于内容的推荐系统}
基于内容的推荐系统是最早被使用的推荐算法,它根据用户过去喜欢的物品,为用户推荐和它过去喜欢的物品相似的物品。
例如,一个推荐电影的系统可以根据某个用户之前喜欢看喜剧电影而为他推荐新出品的喜剧电影。
该算法的核心假设是``一个用户可能会喜欢和他曾经喜欢过的物品相似的物品'',
这里的相似的物品是通过物品的内容属性来确定的。在不同的推荐场景中,物品的内容属性往往是不同的。
电影推荐中,被推荐的物品是电影,其属性包括导演、演员、风格和发布年份等。
音乐推荐中,被推荐的物品是音乐,其属性包括歌手、曲风、语种和音乐时长等。
商品推荐中,被推荐的物品是商品,其属性包括种类、价格、销量和上市时间等。

%优缺点

基于内容的推荐系统的过程一般包括以下三个步骤:
\begin{enumerate}
\item \textbf{物品建模:} 通过物品的内容属性,为每个物品抽取一些特征,来表示该物品。
\item \textbf{用户建模:} 利用用户曾经喜欢及不喜欢的物品的特征信息,加权整合出该用户的特征信息,来表示该用户。
\item \textbf{推荐生成:} 通过比较用户模型和候选物品模型的相关程度,为该用户推荐一组相关度最大的物品。
\end{enumerate}

其中物品的内容属性主要分为两类:结构化的属性和非结构化的属性。
所谓结构化的属性就是这个属性的意义比较明确,其取值限定在某个范围内,
而非结构化的属性意义比较模糊,取值也没有限制,难以直接使用。
以电影推荐为例,电影的导演,演员、风格和发布年份都是结构化属性,而电影的剧情介绍等文本信息就属于非结构化属性。
对于结构化信息,我们可以直接将其加到物品特征向量中,而对于非结构化信息,
一个常见的方法是将其转化为向量空间模型(Vector Space Model),
例如我们可以将电影的剧情介绍的文本信息转化为一个长度为词汇表大小的向量,
该向量的每一个维度表示对应词语的权重,TF-IDF是一个常见的权重表示方法,其公式如下:
\begin{equation}
\begin{split}
TF\textrm{-}IDF(t_i, d_j) = TF(t_i, d_j) \cdot \log{ \frac{N}{n_i} }
\end{split}
\end{equation}

其中$TF(t_i, d_j)$是第$i$个词语在第$j$篇文档中出现的次数,$N$是所有文档的数量,$n_i$是包含第$i$个词语的文档数量。
通常我们需要进行归一化以便将不同文档的向量统一到一个数量级上,其公式如下:
\begin{equation}
\begin{split}
w(i, j) = \frac{ TF\textrm{-}IDF(t_i, d_j) }{ \sqrt{\sum_{k=1}^{T}{TF\textrm{-}IDF(t_k, d_j)^2}} } 
\end{split}
\end{equation}

其中$T$代表所有词语的数量。学术界提出了多种计算用户模型和物品模型相似度的算法,
其中最著名的就是cosine相似度,其公式如下:
\begin{equation}
\begin{split}
sim(d_i, d_j) = \frac{ \sum_k{ w(k, i) \cdot w(k, j) } }
{ \sqrt{\sum_{k}{w(k, i)^2}} \cdot \sqrt{\sum_{k}{w(k, j)^2}} } 
\end{split}
\end{equation}

当然,真正的推荐系统中使用的策略往往可以很复杂,比如在用户建模时考虑时间因素,
计算不同时间段内的用户模型,从而发现用户兴趣在历史数据上表现出的偏好变化。
又比如再推荐生成的过程中,使用决策树、支持向量机和神经网络等更高级的模型,
这些方法最核心的环节就是利用用户模型和物品模型之间的相似度进行计算,
因为基于内容的推荐系统不是本文研究的重点,所以在此不加赘述。

\subsection{基于协同过滤的推荐系统}
协同过滤(Collaborative Filtering)算法是推荐系统中应用最为广泛和成功的算法,与传统的基于内容的推荐系统不同,
协同过滤算法分析用户的兴趣,在用户群中找到给定用户的相似用户,综合相似用户对某一信息的评价,
形成系统对该指定用户对此信息的喜好程度预测。
与传统的基于内容的推荐系统相比,协同过滤算法有下列优点:
1. 能够处理难以获取内容信息的物品推荐,比如艺术品,音乐等。
2. 能保证较高的推荐的新颖性

同时,协同过滤算法也有下列缺点:
1. 当用户对物品的评价数据较为稀疏时,推荐效果较差
2. 随着用户和物品的增多,系统的性能会越来越低
3. 冷启动问题



基于协同过滤的推荐系统可以分为三个子类:基于用户的协同过滤,基于物品的协同过滤和基于模型的协同过滤。
\subsubsection{基于用户的协同过滤}
\subsubsection{基于物品的协同过滤}
\subsubsection{基于模型的协同过滤}
当分解成了Q和P两个矩阵后,空白处的值就可以通过Q矩阵的某列和P矩阵的某列相乘,得到缺失处的值。这背后的核心思想,简单说就是,找到两个矩阵,它们相乘之后得到的那个矩阵的值,和待分解矩阵中有值的位置中的值尽可能地接近,这样以来,分解出来的两个矩阵相乘就尽可能地还原了待分解矩阵,因为有值的地方,值都相差得尽可能地小,那么没有值的地方,通过这样的方式算出的值,也就比较符合趋势,毕竟矩阵尽可能地接近了有值的位置上的值。



\subsection{混合的推荐系统}
由于不同的推荐算法各有利弊,所以在实际的应用场景中,单纯一种推荐策略往往并不能满足推荐需求。
工业界的具体实现中,人们常常将多个方法进行混合,从而达到更好的推荐效果。
关于如何混合不同的推荐策略,这里简单地描述几种比较流行的混合方法。
\begin{itemize}
\item \textbf{加权的混合:}
参考集成学习(Ensemble Learning)的思想,使用若干不同的推荐算法获得多个推荐结果,
然后使用线性结合的方式将这些推荐结果按照一定权重组合起来,最后按照新的权重进行重排序,
产生新的推荐列表。这里的具体权重的数值需要在训练数据集上反复试验,从而达到最好的混合推荐效果。
\item \textbf{分层的混合:}
加权的混合策略可以看成是对多个推荐算法的并联处理,同样我们也可以选择串联处理,
将一个推荐方法的输出作为另外一个推荐方法的输入,以此类推,构成一个串联结构,
从而综合各个推荐策略的优缺点,得到更加准确的推荐结果。
\item \textbf{切换的混合:}
除了将多个推荐模型混合在一起,我们也可以选择在不同的生产环境下选择最适应当前情况的推荐方法。
例如,在商品推荐的场景下,如果发现用户数量超过物品数量,我们可以选择使用基于物品的协同过滤算法,
而当物品数量超过用户数量时,我们切换至基于用户的协同过滤算法。
\item \textbf{分区的混合:}
另外,我们也可以在推荐结果展示界面做文章。采用多种推荐机制,将不同的结果分在不同的区域展示给用户。
例如,亚马逊、淘宝和京东等很多电子商务网站都是采用这样的方式,
用户可以得到很全面的推荐,也会更加容易找到他们想购买的商品。
\end{itemize}


\section{推荐系统的评价}
一个完整的推荐系统一般包含两个参与者:用户和物品。以电影推荐为例,
首先推荐系统需要满足用户的需求,为用户推荐其可能喜爱的电影,
其次,推荐系统也需要满足物品的需求,保证各个电影都有被推荐的机会。
一个良好的推荐系统需要同时考虑两者的利益,构成双方共赢的场面。

在推荐系统中,主要存在三种评价推荐效果的实验方法:离线实验、在线实验和用户调查。

离线实验首先通过日志系统获取用户的行为数据,生成一个标准的数据集,
然后按照一定规则将其切分为训练集和测试集,实验者在训练集上训练推荐模型,
并在测试集上进行预测,最后通过离线评价算法得出推荐模型的预测效果。
离线实验不需要一个实际的系统和真实的用户进行验证,可以进行直接快速的计算,
从而方便进行大量的算法验证,是研究人员的首选方案。
但是离线实验无法得出很多商业模式上关心的点击率和转换率等指标。

在线实验又可以称为AB测试,它通过一定规则将在线用户随机分为两组,
一组采用原始的推荐算法,另外一组使用待验证的推荐算法,
通过统计两组用户的点击率和转化率可以对比待验证算法和原始推荐算法的优劣。
在线测试可以公平地获取不同算法实际在线时的性能指标,
但是其成本开销较大,且必须进行长期的实验才能得到可靠的效果,
不太适用于学术界的研究钻研。

用户调研是随机招募一些真实用户,让他们在待测试的推荐系统上完成一些任务,
实验者观察并记录用户行为,最后让用户回答一些问题,
通过用户行为和用户答案我们可以获知待测试推荐系统的性能。
用户调研需要保证分布真实,比如男女各半,还需要进行双盲实验。
用户调研可以直观得出用户使用过程中的感受,
缺点是很难进行大规模的验证,统计意义不足,容易出现较大的效果偏差。

由于推荐系统应用场景的多样性,历史上研究人员提出了多种评价推荐系统性能的指标,
有些指标可以定量计算,有些只能定性描述。我们下面简单介绍其中有名的几类。

\subsection{用户满意度}
用户满意度是用户对推荐系统效果最直观的评价,其无法通过里县实验计算,
只能通过在线实验或者用户调研获得。
在线实验中,用户满意度一般和用户行为挂钩,
我们既可以在推荐结果列表里添加满意和不满意两种按钮供用户进行直接反馈,
也可以通过点击率,停留时间和转化率等指标进行衡量。
用户调研中,用户满意度一般和用户答案挂钩,
通过设计问卷调查题目,我们可以了解用户对当前推荐系统各个方面的满意程度。

\subsection{预测准确度}
预测准确度是离线实验中最重要的评价指标,其度量推荐系统预测用户未来行为的能力,
主要分为两种:评分预测和TopN预测。

\begin{itemize}
\item \textbf{评分预测}

推荐系统的很多应用场景中,都提供了用户给物品进行评分的功能,
通过评分数据可以反应用户对物品的偏爱程度。
通过用户的历史评分数据,实验者建立推荐模型,预测用户将来的评分数据,
系统系统可以选择其中评分较高的物品进行推荐。
评分预测的预测准确度一般通过均方根误差(Root Mean Square Error, RMSE)
和平均绝对误差(Mean Absolute Error, MAE)计算。
对于测试集T中的每一个用户$u$和物品$i$,假设$r_{ui}$代表用户$u$对于
物品$i$的真实评分,$\hat{r}_{ui}$代表推荐算法的预测评分,
则均方根误差的计算公式为:

\begin{equation}
\begin{split}
RMSE = \sqrt{\frac{\sum_{u,i \in T}{(r_{ui} - \hat{r}_{ui})^2}}{|T|}}
\end{split}
\end{equation}

平均绝对误差的计算公式为:

\begin{equation}
\begin{split}
MAE = \frac{\sum_{u,i \in T}{|r_{ui} - \hat{r}_{ui}|}}{|T|}
\end{split}
\end{equation}

对比均方根误差和平均绝对误差,
Netflix认为均方根误差加大了对预测不准的评分的惩罚(平方项的惩罚),
因此均方根误差相对平均绝对误差更加苛刻。

\item \textbf{TopN预测}

推荐系统的展示结果一般是一个包含若干物品的个性化推荐列表,我们一般称之为TopN推荐,
其预测准确率一般通过对比个性化推荐物品列表和用户真实喜爱物品列表,
计算准确率(Precision)和召回率(Recall)来实现。
对于测试集T中的每一个用户$u$,假设$R(u)$代表用户$u$真实喜爱的物品列表,
而$\hat{R}(u)$代表推荐给用户$u$的个性化推荐列表,
则准确率的计算公式为:

\begin{equation}
\begin{split}
Precision = \frac{ \sum_{u \in T}{|R(u) \cap \hat{R}(u)|} }{ \sum_{u \in T}{|\hat{R}(u)|} }
\end{split}
\end{equation}

召回率的计算公式为:

\begin{equation}
\begin{split}
Recall = \frac{ \sum_{u \in T}{|R(u) \cap \hat{R}(u)|} }{ \sum_{u \in T}{|R(u)|} }
\end{split}
\end{equation}

\end{itemize}

准确率体现了在我们推荐的物品列表里面,用户喜爱的占比多少。
而召回率体现了在用户喜爱的物品列表里面,我们推荐的占比多少。
两者都是数值越大,推荐效果越好。
同时,我们也会使用$F1$值来结合准确率和召回率,其计算公式为:

\begin{equation}
\begin{split}
F1 = \frac{Precision \cdot Recall}{ \alpha \cdot Precision + (1 - \alpha) \cdot Recall }
\end{split}
\end{equation}

其中,$\alpha$的数值位于0到1之间,用于控制准确率和召回率两者相对的影响因子。

\subsection{覆盖率}


\subsection{其他一些指标}







