\chapter{章节}
\section{研究背景}
    近年来,微博作为一个方便快捷的信息发布和分享平台,受到了广大用户的青睐,越来越多的用户喜欢在微博平台中发布和分享自己感兴趣的信息。
    根据统计,在著名微博平台推特(Twitter)中,有289,000,000个活跃用户,每天平均产生58,000,000条微博信息\footnote{http://www.statisticbrain.com/twitter-statistics/}。
    相比于传统的新闻媒体,微博等社交媒体具有两方面的优势。首先是信息更新和传播速度快,用户可以第一时间了解到事情的发展动向;其次,微博的交互式信息传播方式能够让更多的用户通过转发和评论等方式参与信息的分享和互动。
    由此可以看到,微博已经融入到人们的生活中,并且已经成为了一个重要的信息资源。很多用户会在微博中了解自己感兴趣的话题或者热门的事件,比如苹果新出的手机或者Google AlphaGo挑战李世石等等话题和事件。
    从微博媒体中用户可以了解到很多重要的相关信息,比如其他用户的评价和看法以及官方媒体的实时报道和事件更新等等。
    但在享受着微博遍历的信息渠道的同时,由于用户的时间和精力有限,数量庞大的微博流很可能导致一个严重的信息过载问题,即用户很难方便快捷地找到自己感兴趣的信息。因此,本文研究了自适应的微博实时过滤技术,能够根据用户的感兴趣的话题给用户实时推送相关的微博。
